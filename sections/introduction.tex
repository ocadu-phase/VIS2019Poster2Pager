In medicine, doctors rely on visualization tools to display tumor or any other lesions, in a clear manner such that the spatial relationship to anatomical structures of the brain are perceived.  Typically, they use magnetic resonance images (MRI) to obtain volumetric data and display them onto a monitor.  They manipulated the 3D data, inherent to MRI, as 2D data through a monitor.  These visualization tools often show various angles of the data so that the physician can infer the relative position, orientation, shape and relevant anatomical cues to disambiguate the spatial relationship between the tumor and the rest of the brain structure.\\

We hypothesize that viewing 3D structured data is best viewed using true 3D displays such as holograms. We believe that it can be more effective in representing the tumors shape and position especially in situations when the shape is abnormal and the tissue is statistically similar to its surrounding.  Furthermore, holograms has an additional advantage to other 3D displays in that  no special viewing aids are needed such as special glasses or a head worn display such as a VR headset.  \\

In addition because holograms are in colour we can utilize different color mapping technique to highlight various regions in the MRI without manipulating the data, which is a necessity in the medical field.  Furthermore, in the past depth was used as a colour parameter but using holograms depth is inherent so we can use the raw data as parameters for colour instead. In the literature it has been established that with the appropriate colour mapping techniques colormaps play an important role in visualization because they are able to improve the efficiency and effectiveness of data perception which in turn allow more insights into the data.\\  

The main focus and contribution of this poster is to highlight the importance of colour as well as the visualization medium to optimally view medical volumetric data such as MRIs, and provide an alternative visualization pipeline to viewing them.

