The current approach that our group has been using to generate medical holograms is first constructing a model using segmentation. For this we employ an intuitive approach of pixel classification using multi-thresholding  techniques in combination with region growing approaches.  Following this step we perform computer graphic rendering, and hologram encoding.  There are three main issues with pipeline.  The first being a non-automatic method which involves human intervention, second is that it is computationally expensive, and the third is that it generates occlusion between the geometry in the scene.\\

Occlusion is an important and powerful cue to scene layout, and based on psychology studies it usually caries the precept even in the presence of conflicting information.  Analogous to colour, the rendering approach should also report the occlusion relationship between objects within the scene. In this diagnostic application it is important not to hide information of potential interest.  As a result, the semi-transparent and non-occluding nature of volumetric rendering maybe appropriate for some medical diagnostic applications.\\

Volume rendering (technically known as direct volume rendering) in the field of computer graphics is considered to be a form of 3D scalar field.  Its a 3D random density plot composed of spheres that pertain to a particular colour, that essential emit light without any reflectance.  Our approach uses a texture-based volumetric rendering to generate the views of the medical images, capable of being handled by powerful graphics cards yielding quick results and real-time interaction.\\

Combining this approach with the results from luminance distribution tests with colormaps, we create a multivariate workflow of exploring volume data and representing high quality views needed for analysis.  3D volume segmentation techniques are further explored applying luminance distribution in other ways including using a dual colormap strategy in our volume rendering for greater lighting control, and shifting the order of image processing tasks such as applying colourmaps to independent slices versus the direct volume.\\
