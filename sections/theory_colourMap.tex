In the literature as well as in most scientific visualization programs, colour has been proven to be an effective tool.  However, no single color-map is optimized for all science domains.  Current default colour schemes such as the rainbow, jet, gray-scale and cool to warm often obscure data and fail to solve the occlusion relationships in volumetric rendering of data.  There have been work in this field to combat this issue and user-study-based, rule-based and data-driven methods have been explored.  Whoever we believe that the work done by [Samel] is the most effective method in solving this occlusion problem.  We extend their work into the 3D medical field by applying two principle from artistic colour theory: maximizing values contrast and avoiding simultaneity of colour.\\

Humans ability to focus attention and discriminate detail is governed by the type and degree of contrast and not by specific hues.  In our case-study of using a human brain and tumor, it is essential for the tumor to be apparent and salient in the scene.  One major pitfall of the current colour schemes is the visual vibration caused by the abutting saturated hues of the colourmaps.  As a result our mapping system will focus more on the allocation of contrast, specifically luminance and saturation. Hence we use the following colour schemes:\\

MARCUS TO ADD THE COLOUR MAPS WE ARE USING!!!

